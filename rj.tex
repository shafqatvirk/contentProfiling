\documentclass[12pt]{article}
\usepackage[authoryear,round,comma,sort]{natbib}
\usepackage{url}

\author{Markus Forsberg, Harald Hammarstr\"om and Shafqat Virk}
\title{Shaf-Cat: Capturing Language Diversity Through Raw-Text Induced Language Profiles}

\begin{document}
\maketitle

\section{Summary}

Sammanfattningen ska skrivas på svenska och engelska och får omfatta max 1 500 tecken inklusive blanksteg och ska förklara varför forskningsuppgiften är viktig, dess syfte samt hur den ska genomföras. Den ska formuleras så att en intresserad allmänhet kan förstå. Om ansökan beviljas publiceras sammanfattningarna direkt på RJ:s webbplats.

\section{Project Description}
The diversity of the 6,500 languages of world represent an
irreplaceable and abundant resource for understanding the unique
communication system of our species. Rather than to study just one
language, such as English, through the comparison of many languages we
are better equipped to trace the history of the populations that speak
them as well as to understand the processing machinery of our brains
(cf.~\citealt{typ:EvansLevinson:Universals}).

For most of its history, comparison of languages has been carried out
by humans qualitatively. This research framework is naturally bounded
by the limits of human capacities, and as such can only target a few
dozen languages in detail and/or target greatly simplified
characteristics of more languages. We are now entering a phase where
it is practical to use computers for language comparison, entertaining
the potential to compare languages in much greater breadth and
depth. However, the databases used for computational comparisons are
still designed and curated manually by humans who read grammars and
dictionaries in book form, and herein lies the bottleneck. Firstly, as
a language is a system of conventional form-meaning pairs, it is not
obvious how to represent it in machine-readable form. Especially when
it comes to grammar (as opposed to phonology or lexicon), theory is
lacking on what, how and to what granularity it should be rendered
(cf.~\citealt{ling:Cysouw:Typology-Types}). As a result, linguistic
typological databases used today reflect collections of properties
traditionally of interest, but without systematic motivation in terms
of in terms of functional load, role in the language system or the
like \citep{ling:Bickel:Distributional-Typology}.  Secondly, manual
data collection by humans is a time-expensive enterprise --- a
database treating a single linguistic topic for some 200 languages is
typically the size of a PhD project, whereas the world has
6,500 languages and there is grammatical information for over 4,000
(see \url{glottolog.org}).
 
The present project aims to fill this gap, in three steps:
\begin{description}
\item[Language Profiles:] The development of a theoretically motivated
  notion of a \emph{language profile}, i.e., a machine-readable
  representation of features of the grammar of a language. This notion
  will be grounded in the functional load of various parts of the
  language system and parametrized by a single scale indicating the
  level of detail, whereby a low level of detail results in a single
  feature profile, a relatively high level of detail results in a
  profile with many features and an even higher levels of detail
  approximates a language 'in its full glory'.

\item[Profile Extraction from Raw Text:] A large collection of freely
  available grammatical descriptions of languages of the world
  (see below)
  (TODO cite etc) have been OCR:ed and are ready for NLP analysis.
  Blah blah 

\item[Application:] Two concrete case-studies related to linguistic
  diversity will be performed within the realm of the project as a
  proof-of-concept, one targeting population history and one targeting
  language-universal tendencies. For the first, extracted language
  profiles will be used to (in)validate the macro-areal division of
  the (typological profiles of the) languages of Africa
  \citep{hv:Guldemann:Africa:Macro-Areas} and improve on it in detail
  and coverage. For the second, extracted information relating to
  constituent- and morpheme ordering properties will be compared to
  data on the 'suffixing preference'
  \citep{ling:Himmelmann:Suffixing:2014} and the extensive study of
  word order by \citet{ling:Dryer:Word-Order}, also improving them on
  detail and coverage.
\end{description}

The basis for the entire project is a collection of over 6,000 raw
text grammatical description digitally available for computational
processing. The collection consists of (1) out-of-copyright texts
digitized by national libraries, archives, scientific societies and
other similar entities, and (2) texts posted online with a CC-BY-SA (or
more liberal) license usually by university libraries and non-profit
organizations (notably the Summer Institute of Linguistics). For each
document, we know the language it is written in (the meta-language,
usually English, French, German, Spanish or Mandarin Chinese), the
language(s) described in it (the target language, typically one of the
thousands of minority languages throught the world) and the type of
description (comparative study, description of a specific features,
phonological description, grammar sketch, full grammar etc). The
collection can be enumerated using the bibliographical- and metadata
is contained in the open-access bibliography of descriptive languages
data at \url{glottolog.org}.

The project applicants are committed to open-access framework and the
data emanating from the present project will be deposited in long-term
repositories at Spr\aa{}kbanen and \url{zenodo.org}.

Creating language profiles through the mining of raw text data is a
novel task so specific related work only consists of a few embroyonic
steps taken by the applicants
(\citealt{typ:Hammarstrom:Three-Approaches},
\citealt{cl:Virk:TextCat}). However, the maturity of vector space
semantic representations from raw text data
\citep{cl:Mikolov:Words-Phrases}, OCR correction techniques (TODO
ref), and the availability of evaluation data makes this project both
feasible and evaluatable.

In the realm of lexicon, several projects are in the process of
digitizing legacy printed dictionaries, e.g., RefLex
\citep{v:Segerer:RefLex}, PanLex (\url{www.panlex.org}), Asia-Pacific
Data Warehouse \citep{ling:Cooper:Warehouse}, into structured
databases. The representation of a lexicon is already a
well-understood task so these projects have the character of
engineering projects whereas for the proposed project, the
representation of grammar is the major research question.  A different
promising approach to extracting language profiles is to depart from
Interlinear Glossed Text (IGT) rather than raw text or raw descriptive
text \citep{lkl:Bender:From-IGT}. Such an approach is also on a sound
theoretical footing but collections of IGT text, e.g., ODIN
(\url{depts.washington.edu/uwcl/odin/}), ELAR
(\url{http://elar.soas.ac.uk/},) DOBES (\url{dobes.mpi.nl/}) do not
yet span enough languages and suffer from heterogeneities and
usage/access restrictions.

The potential for descriptive text extraction for grammatical
comparison cannot cannot be underestimated; When successful it will
transform the wealth of knowledge that was until recently (largely)
collecting dust on the shelves into a data source multiplying tenfold
the breadth and depth of under the command of the linguist.
Beyond linguistics itself, the concept may also generalize to other
areas of the humanities, such as religion or ethnography, who are
engaged in cross-cultural comparison.


\section{Project Plan}

Projektbeskrivningen får omfatta max 9 600 tecken inklusive blanksteg, skrivas på engelska eller svenska och redovisa:


\section{Researchers' Present Function in the Project, Relevant Skills and Merits, and Current Employment Situation}

\subsection{Shafqat Virk}
Shafqat Virk at 75\% FTE will perform the extraction of profiles from
raw text data and its evaluation including (shallow) parsing, latent
semantic indexing and other techniques from the NLP toolchain. Shafqat
is currently a PostDoc at Spr\aa{}kbanken (Gothenburg University) in
the LSI project TODO (what's it called officially?). He has a genuine
background in Computational Linguistics and a thorough experience in
the task of extracting grammatically relevant data from raw text
descriptions in the latter project.

\subsection{Harald Hammarstr\"om}
Harald Hammarstr\"om at 50\% FTE will be responsible for the
development of the language profile notion in aided by the text
technology results by the other collaborators. He will also carry ot
the tasks in the application phase. Harald is an Associate Lecturer at
the Department of Linguistics and Philology (Uppsala University) and
is an expert on language diversity and linguistic typology.

\subsection{Markus Forsberg}
Markus Forsberg at 25\% will adapt recent methods from the field of
Deep Learning for NLP to enhance the semantic component in the profile
extraction chain. He will also integrate relevant morphological and
lexical resources and handle data archiving matters. Markus is
currently co-director of Spr\aa{}banken (Gothenburg University). He
has a solid background in Computational Linguistics and a decade of
experience in the fields of text technology, information extraction,
knowledge representation and linguistic resources.

\subsection{Lars Borin?}
TODO

\section{Budget}
Budgeten

    Löner, LKP, drift samt arbetstid för icke-forskande personal ska specificeras. Uppge däremot inte indirekta kostnader och lokalkostnader.
    Namnge all forskande personal. Lista dem som projektdeltagare med månadslön, LKP och arbetstid inom projektet.
    Ange 0 kronor för projektdeltagare som inte avlönas inom projektet.
    Redovisa även arbetstid och lönekostnader för eventuell icke-forskande personal, såsom assistenter och teknisk personal som Drift, men namnge dem inte.
    Till varje beviljat projekt adderas ett bidrag till indirekta kostnader och lokalkostnader enligt den modell som RJ tillämpar.
    Till varje projektanslag beviljar RJ även ett internationaliseringsbidrag. Bidraget kan användas för forskningsvistelser utomlands, inbjudan av utländska forskarkolleger, såväl nationella som internationella konferensresor och språkgranskning av manus på främmande språk. Ta därför inte upp dessa kostnader i din ansökan.
    Till varje projektanslag beviljar RJ även ett bidrag till publicering med open access. Inte heller dessa kostnader ska därför tas upp i ansökan.

Ospecificerade eller omotiverade kostnader beviljas inte.

\bibliographystyle{apalike}
\bibliography{hh,hhling,miscbooks}


\end{document}


\section{Research Question}

\section{Purpose}

\section{Theory and Method}

\section{Contributions to the Scientific Frontier}

\section{Most Important Publications of the Research Area}


    forskningsfrågan
    syftet
    teori och metod
    projektplan och motiveringar till projekttiden
    bidrag till den internationella forskningsfronten
    forskningsområdets internationellt viktigaste arbeten
    forskarens/forskarnas funktion i projektet samt relevanta kompetenser och meriter
    nuvarande anställningsförhållanden för all forskande personal
    resursbehov (löner och drift) samt arbetstid för icke-forskande personal som specificeras och motiveras



Med projekt avses väl definierade forskningsuppgifter inom samhällsvetenskap och humaniora som utförs av såväl enskilda forskare som av mindre forskargrupper med anknytning till Sverige. Forskarna är fria att själva definiera sin forskningsfråga. Projekt beviljas för maximalt tre år.



Bedömningskriterier

De avgörande kriterierna vid RJ:s beslut är den vetenskapliga kvaliteten i forskningsinnehållet, projektets genomförbarhet samt forskningens potentiella bidrag till internationell forskning.
Beredningsprocess

Ansökningsprocessen sker i två steg. Projekt bedöms och prioriteras i en eller flera av RJ:s beredningsgrupper som består av svenska och internationella forskare samt några av styrelsens riksdagsledamöter. De projekt som väljs ut för fortsatt beredning skickar in en utvecklad ansökan som sedan granskas av externa sakkunniga från Sverige eller utlandet, vars utlåtanden ligger till grund för beredningsgruppernas bedömning i andra omgången.

Stiftelsens styrelse fattar det formella beslutet om forskningsanslagen. Skriftliga utlåtanden från de externa sakkunniga skickas till alla som har gått vidare till andra omgången efter beslutssammanträdet i oktober.
Första omgången

Redan i första ansökningsomgången ska projektledaren informera prefekt eller motsvarande om att medel söks hos RJ.

Sammanfattningen ska skrivas på svenska och engelska och får omfatta max 1 500 tecken inklusive blanksteg och ska förklara varför forskningsuppgiften är viktig, dess syfte samt hur den ska genomföras. Den ska formuleras så att en intresserad allmänhet kan förstå. Om ansökan beviljas publiceras sammanfattningarna direkt på RJ:s webbplats.

Projektbeskrivningen får omfatta max 9 600 tecken inklusive blanksteg, skrivas på engelska eller svenska och redovisa:

    forskningsfrågan
    syftet
    teori och metod
    projektplan och motiveringar till projekttiden
    bidrag till den internationella forskningsfronten
    forskningsområdets internationellt viktigaste arbeten
    forskarens/forskarnas funktion i projektet samt relevanta kompetenser och meriter
    nuvarande anställningsförhållanden för all forskande personal
    resursbehov (löner och drift) samt arbetstid för icke-forskande personal som specificeras och motiveras

Budgeten

    Löner, LKP, drift samt arbetstid för icke-forskande personal ska specificeras. Uppge däremot inte indirekta kostnader och lokalkostnader.
    Namnge all forskande personal. Lista dem som projektdeltagare med månadslön, LKP och arbetstid inom projektet.
    Ange 0 kronor för projektdeltagare som inte avlönas inom projektet.
    Redovisa även arbetstid och lönekostnader för eventuell icke-forskande personal, såsom assistenter och teknisk personal som Drift, men namnge dem inte.
    Till varje beviljat projekt adderas ett bidrag till indirekta kostnader och lokalkostnader enligt den modell som RJ tillämpar.
    Till varje projektanslag beviljar RJ även ett internationaliseringsbidrag. Bidraget kan användas för forskningsvistelser utomlands, inbjudan av utländska forskarkolleger, såväl nationella som internationella konferensresor och språkgranskning av manus på främmande språk. Ta därför inte upp dessa kostnader i din ansökan.
    Till varje projektanslag beviljar RJ även ett bidrag till publicering med open access. Inte heller dessa kostnader ska därför tas upp i ansökan.

Ospecificerade eller omotiverade kostnader beviljas inte.


Sök bidrag till konferenser, workshops och nätverk som du själv vill arrangera via stödformen Forskningsinitiering.

Approximately 6,500 mutually unintellible languages constitute our
record of linguistic diversity
\citep{h:Hammarstrom:Ethnologue:161718}. Languages are equal witnesses
--- where e.g., English is but one --- to the variation and
constraints of the unique communication system of our species,
\citep{typ:EvansLevinson:Universals}. They hoard information
information on what happens to language given tens of thousands of
millenia of diversification, under all imaginable circumstances of
human interaction. As such they may be used to investigate theories
that may otherwise not be testable with anything less than a
laboratory the size of human history.


